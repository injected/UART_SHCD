%
% Glossar
%
\newglossaryentry{AIISO}{
name={Academic Institution Internal Structure Ontology (AIISO)},
description={Ontologie für die Beschreibung der Struktur akademischer Einrichtungen}
}

\newglossaryentry{CSV}{
name={Character Separated Values (CSV)},
description={Format bei dem die einzelnen Elemente durch spezielle Zeichen, z.B. Strichpunkte, getrennt werden}
}

\newglossaryentry{DC}{
name={Dublin Core (DC)},
description={Sammlung von Metadatenelementen zu Beschreibung von Dokumenten/Ressourcen}
}

\newglossaryentry{DAML}{
name={DARPA Agent Markup Language (DAML)},
description={Auszeichnungssprache für das Semantic Web}
}
%\newacronym{EBNF}{EBNF}{Erweiterte Backus-Naur-Form\protect\glsadd{glos:EBNF}}
\newglossaryentry{EBNF}{
name={Erweiterte Backus-Naur-Form (EBNF)},
description={Formale Metasyntax zur Darstellung kontextfreier Grammatiken}
}

\newglossaryentry{F-Logic}{
name={Frame Logic (F-Logic)},
description={Formale Sprache zur Formulierung von Ontologien}
}

\newglossaryentry{FOAF}{
name={Friend of a Friend (FOAF)},
description={Ontologie formalisiert in RDF-Schema für maschinenlesbare Modellierung sozialer Netzwerke}
}

\newglossaryentry{INKIDU}{
name={INKIDU},
description={Inkidu ist ein an der HS Ravensburg-Weingarten entwickeltes Internet Umfragesystem},
first={INKIDU},
text={}
}

\newglossaryentry{HIS}{
name={Hochschul-Informations-System (HIS)},
description={Softwarehaus mit Sitz in Hannover, Entwickler des LSF}
}

\newglossaryentry{KOS}{
name={Knowledge Organization Systems (KOS)},
description={Ein KOS oder Begriffssystem fasst Begriffe oder Klassen aus abgetrennten System zusammen.}
}

\newglossaryentry{LSF}{
name={Lehre, Studium, Forschung (LSF)},
description={Verwaltungsportal für Hochschulen mit Stundenplänen, Prüfungsplänen uvm.}
}
 
\newglossaryentry{Metadaten}{
name={Metadaten},
description={Daten über Daten, Daten welche den Inhalt näher beschreiben},
first={Metadaten},
text={}
}

\newglossaryentry{Microformats}{
name={Microformats},
description={Microformats ist ein Markup-Format für die semantische Beschreibung von (X)HTML Webseiten},
first={Microformats},
text={}
}

\newglossaryentry{OIL}{
name={Ontology Inference Layer (OIL)},
description={Webbasierte Auszeichnungssprache für das Semantic Web}
}

\newglossaryentry{Ontologie}{
name={Ontologie},
description={Menge von Begrifflichkeiten und deren Beziehungen für eine bestimmte Domäne},
first={Ontologie},
text={}
}

\newglossaryentry{OWL}{
name={Web Ontology Language (OWL)},
description={Beschreibt Zusammenhänge zwischen Termen in RDF}
}

\newglossaryentry{Pattern-Matching}{
name={Pattern-Matching},
description={Sucht nach einem bestimmten vorgegebenen Muster in einer diskreten Struktur},
first={Pattern-Matching},
text={}
}

\newglossaryentry{RDF}{
name={Resource Description Framework (RDF)},
description={Framework für die Beschreibung von Objekten (Ressourcen) und deren Eigenschaften in Triples}
}

\newglossaryentry{RDFS}{
name={RDF-Schema (RDFS)}, description={Erweitert RDF und formalisiert einfache Ontologien für eine Domäne}
}

\newglossaryentry{REST}{
name={Representational State Transfer (REST)}, description={Architektur für verteilte Anwendungen, beispielhaft dafür ist das World Wide Web}
}

\newglossaryentry{RESTful}{
name={RESTful}, description={Erfüllt die Anforderungen an einen REST Webservice}
first={RESTful},
text={}
}

\newglossaryentry{SKOS}{
name={Simple Knowledge Organization System (SKOS)},
description={Auf RDF basierende Sprache zur Formalisierung von Klassifikationen und Thesauri}
}

\newglossaryentry{SPARQL}{
name={SPARQL Protocol and RDF Query Language (SPARQL)},
description={Graph-basierte Abfragesprache für RDF}
}

\newglossaryentry{triple store}{
name={triple store},
description={Speichert RDF Metadaten in einer Art Relationalen Datenbank}
first={triple store},
text={}
}

\newglossaryentry{URI}{
name={Uniform Resource Identifier (URI)},
description={Identifikator um jegliche Ressource im WWW eindeutig zu identifizieren}
}

\newglossaryentry{W3C}{
name={World Wide Web Consortium (W3C)},
description={Standardisierungsgremium für das World Wide Web}
}

\newglossaryentry{WOT}{
name={Web of Trust (WOT)},
description={Mit Hilfe einer Community wird ein Vertrauensnetzwerk aufgebaut, um Webseiten und Inhalte als nicht gefährlich einzustufen und um Personen zu authentifizieren.}
}

\newglossaryentry{XML}{
name={XML},
description={eXtensible Markup Language, Auszeichnungssprache für hierarchische Daten}
}
