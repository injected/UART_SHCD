\chapter{Einleitung}
\label{cap-einleitung}
\section{Motivation und Zielsetzung} % karl
%Die Aufgabenstellung im Fach \textit{Informationsmanagement mit XML} war es, eine Beispielanwendung mit Hilfe von semantischen Web Hochschulwebseiten (klingt kacke, weglassen??)

Über das Semantic Web gibt es bereits zahllose theoretische Arbeiten und Bücher. Auch an der Hochschule Ravensburg-Weingarten wurde in den letzten Jahren mehr über das Semantic Web geschrieben, als dass eine Projektgruppe versucht hat diese Technologien in einem Projekt praktisch umzusetzen. Daher ist die praktische Betrachtung der Technologien für uns besonders von Bedeutung. Dies manifestiert sich neben dem Titel dieser Ausarbeitung auch in der Aufteilung des Dokuments.

Neben dem kurzen Kapitel Grundlagen (\ref{cap-grundlagen}) stellen wir direkt die Projektidee vor (\ref{cap-projektidee}). Im Verlauf dieses Kapitels folgt eine Analyse der Idee im Bezug auf benötigte und verfügbare Daten sowie deren Formate (\ref{sec-analyse-daten}). Im Abschnitt \ref{sec-open-linked-data-hs} wird erläutert wie es um die Verfügbarkeit von semantischen Daten an der Hochschule bestellt ist. Dies ist deswegen so wichtig, weil es sich direkt auf das Ziel unseres Projektes auswirkt. Der Abschnitt Analyse skizziert die \gls{Ontologie} (\ref{sec-idee-datenmodel}) für die Hochschuldaten und anschließend rückt der Fokus auf die Umsetzung dieser Idee. In der Umsetzung (\ref{cap-umsetzung}) werden Toolkits und Frameworks zum einfachen Arbeiten mit Sematic Web Technologien im Einsatz beschrieben (\ref{sec-daten-python} und \ref{sec-daten-joseki}). Abschließend fassen wir unsere Ergebnisse zusammen (\ref{sec-fazit}) und geben einen kurzen Ausblick (\ref{sec-ausblick}).

%, der versucht vom konkreten Projekt den Bogen auf die allgemeine Entwicklung des Semantic Web zu spannen.
%Beleuchtet wird neben den Programmen und Sprachen () auch die Hürden im Projekt (TODO), vor allem im Bezug auf die Verfügbarkeit von Daten.

\section{Arbeitsverteilung}

\begin{table}[h!]
\centering
\begin{tabular}{|l|l|l|}
\hline \textbf{Name} & \textbf{Kapitel/Abschnitt} & \textbf{Weitere Aufgaben} \\ 
\hline Benjamin Merkle & \ref{cap-grundlagen}; \ref{klassifizierung}; Glossar & Diagramme  \\ 
\hline Karl Glatz & \ref{cap-einleitung}; \ref{sec-grundlagen-einf-semweb}; \ref{cap-projektidee} & Konzeption, Konfiguration, JavaScript \\  %Koordination %(\ref{sec-idee-datenquellen} \ref{sec-idee-hintergrund})
\hline Sebastian Wiedenroth & \ref{sec-sparql}; \ref{sec-idee-datenmodel}; \ref{cap-umsetzung} & Konzeption, Python Code, JavaScript \\ 
\hline Andreas Kimpfler & Abstract; \ref{sec-algo}; \ref{cap-zusafa} & Dokument-Style, Server, Dummy Daten \\  %\ref{cap-zusafa}
\hline 
\end{tabular}

\caption{Arbeitsverteilung der Projektteilnehmer}
\end{table}

