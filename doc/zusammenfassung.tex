%
% Zusammenfassung
%
\chapter{Zusammenfassung}
\label{cap-zusafa}

%\todo[inline]{In einleitung schauen was hier genau hinkommt...}

\section{Fazit}
\label{sec-fazit}
Die Erfahrungen mit der vom Team erstellten Anwendung zeigt deutlich, dass es zwar ungewohnt, jedoch nicht allzu kompliziert ist, Daten semantisch aufzubereiten und zu verarbeiten. Anwendungsentwicklung mit Semantic Web Techniken ist einfach und kann viel Spaß bereiten. Alle Projektteilnehmer haben viel gelernt.

Positiv überrascht wurden wir durch die Flexibilität und Kooperationsbereitschaft verschiedener Instanzen.
Hierbei möchten wir uns noch einmal beim Rechenzentrum der HS-Weingarten und Marius Ebel vom INKIDU-Team für die Zusammenarbeit bedanken. Das Rechenzentrum hat uns den, für die Entwicklung den nötigen (virtuellen) Server, bereitgestellt. Hierfür möchten wir uns ebenfalls bedanken.

Aufgrund des kurzen zeitlichem Rahmens dieser Arbeit konnten leider nicht alle Ziele erreicht werden.
Auf die Daten des LSF System mussten wir leider verzichten, da der Export vom Rechenzentrum so kurzfristig nicht umgesetzt werden konnte.

Ebenso konnten wir die in dieser Arbeit erstellten Daten nur über Umwege veröffentlichen.
Die sehr restriktiven Firewall-Regeln des Rechenzentrums erschwerten das Bereitstellen der Anwendung unter der Internet-Adresse der Hochschule und war nur über einen externen Proxy möglich.


%  vielleicht anderer Algorithmus zur Bewertung/Ergebnisberechung da Klassifizierungs algorithmen für viele klasse ineffizient werden -> gründ ertel buch S. 198 -> kleinste Quadrate usw.
\section{Ausblick}
\label{sec-ausblick}
Ein Ziel für zukünftige Semantic Web Projekte an der Hochschule Ravensburg-Weingarten muss sein, das Angebot an Daten zu erhöhen und diese bereitzustellen. Vor allem sollten Daten direkt aus dem LSF exportiert werden, da es sich bei dieser Anwendung um die zentrale Dokumentationsquelle aller relevanten Daten der Organisationsstruktur der Hochschule handelt.\\
Würde die Hochschule einen zentrales System für semantische Datenverarbeitung bereitstellen, welcher z.B. unter data.hs-weingarten.de auch aus dem Internet erreichbar ist, so könnte man alle gewünschten Daten semantisch aufbereitet der Öffentlichkeit zur Verfügung stellen. Zudem wäre die Hochschule hier ein Vorreiter in der Nutzung von semantischen Web-Techniken und dies würde langfristig dazu führen, dass zum einen weitere Forschungsprojekte, gegebenenfalls auch aus der Industrie, entstehen, welche auch Geld für selbige mitbringen. Zudem würde es den Ruf der Hochschule, gerade im Bereich Informatik nochmals erhöhen und langfristig auch dafür sorgen, dass andere Hochschulen und Universitäten in Daten semantisch veröffentlichen. Aus diesen Daten, welche dann deutschlandweit von allen Hochschulen existieren ließen sich dann Vergleiche anstellen, Rankings generieren uvm. Den Möglichkeiten sind hier sogut wie keine Grenzen gesetzt.\\
Ein weiterer Aspekt dieser Veröffentlichung wäre, dass dann jeder auf Basis dieser Daten eigene Anwendungen erstellen kann, die z.B. den Studenten das Studium vereinfachen.

Das Projekt ist als OpenSource Projekt im Web\footnote{\url{https://frucman.frubar.net/info-xml/}} verfügbar. Interessierte können an dem Projekt auch nach Ablauf der Projektphase teilhaben.


