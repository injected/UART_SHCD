\chapter*{Abstract}

\begin{abstract}
\textsl{Es gibt viele Ansätze und Forschungsprojekte, die das Semantic Web voran\-treiben wollen. Dennoch gehören Technologien des Semantic Web längst nicht zum Alltagswerkzeug eines Webentwicklers. Diese Arbeit gibt einen Überblick über einige Techniken und verwendet diese im Rahmen einer Beispiel-Anwendung zur Evaluierung von Praktikumsstellen für Studenten des Studiengangs Angewandte Informatik. In der Anwendung kommen verschiedene Techniken aus dem Semantic Web Umfeld zum Einsatz. Die für die Anwendung benötigten Daten werden dabei aus internen Datenbanken, dem Umfragesystem, sowie dem Verwaltungsportal für die Vorlesungen der Hochschule entnommen und aufbereitet. Die gesammelten Daten werden semantisch Aufbereitet im Web zur (freien) Verfügung gestellt. Die von uns entwickelte Ontologie versucht möglichst große Teile standardisierter Ontologien zu verwenden. Die Beispiel-Anwendung verwendet Methoden aus der künstlichen Intelligenz um die Daten zu Klassifizieren.}
\end{abstract}

\selectlanguage{english}
\begin{abstract}

\textsl{There are many approaches and research projects that intend to push the Semantic Web. However semantic web technologies are not yet part of the average web developers toolset. This paper gives an overview of some techniques and uses them as a part of an example application for the evaluation of internships for students studying applied computer science. Different semantic web techniques are used in the application. The data needed for the application comes from internal databases, the survey system, and in addition from the management portal for the lectures of the university and get prepared for further use. We publish the resulting semantic data to the world wide web for free use. The developed ontology uses well known ontology standards where possible. The application uses methods from artificial intelligence to classify the data.}
\end{abstract}

\selectlanguage{ngerman}


